\chapter{Kurzfassung}

\begin{german}
Obwohl das dynamische Typ-System von JavaScript in vielen Szenarien von Vorteil sein kann, erhöht es das Risiko Fehler einzuführen. Um dem entgegenzuwirken, wurde das JavaScript Superset TypeScript entwickelt, welches die Möglichkeit bietet, optional Typ-Annotationen einzufügen, was in erhöhter Leserlichkeit, Skalierbarkeit und Wartbarkeit des Codes resultiert. Zusätzlich können die statischen Typ-Überprüfungen, die vom TypeScript Kompilierer durchgeführt werden, eine Vielzahl an Zuständen erkennen, welche zur Laufzeit des Zielcodes möglicherweise zu Problemen führen können. Obwohl Typ-Kompatibilität zur Zeit der Kompilierung überprüft wird, sind die Typ-Informationen zur Laufzeit nicht verfügbar. Das entfernen dieser Information is beabsichtigt und in den Design-Zielen der Programmiersprache festgelegt. Daher müssen umfangreiche Typ-Prüfungen manuell durchgeführt werden, was in erhöhtem Entwicklungsaufwand und höherer Fehleranfälligkeit resultiert. Die Tatsache, dass geeignete Typ-Informationen für die meisten Situationen verfügbar sind, suggeriert, dass die Generierungen von Laufzeit Typ-Überprüfungen, basierend auf den vorhandenen Daten während der Kompilierung, technisch möglich ist. Die Informationen, welche vom TypeScript Kompilierer entfernt werden, sollen im Zielcode reflektiert werden, um sie für Typ-Kompatibilitätsprüfungen während der Programmausführung zu verwenden. Diese Überprüfungen sollen automatisch generiert und in dem resultierenden JavaScript-Code eingefügt werden, was während der Entwicklung eines Projekts helfen soll, mögliche Problemen zu identifizieren, die der TypeScript Kompilierer nicht feststellen kann.
\end{german}
