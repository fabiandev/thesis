\chapter{Abstract}


Even though JavaScript's dynamic type system can be of advantage in many scenarios, it adds the risk of introducing errors. Therefore TypeScript was defined as a superset of JavaScript, with the ability to optionally add type annotations, resulting in increased code readability, scalability and maintainability. In addition, TypeScript's static compile time type checks can detect a multitude of conditions that may cause issues in the target code at runtime. Although type compatibility is checked during compilation, type information is not available in the compiled JavaScript code, i.e., at runtime. The removal of the type information is intended and is defined in the design goals of the language, although dynamic type validations can improve the quality of the executable program. By extending the TypeScript compiler, this thesis provides an efficient method to maintain the type information for type checks at runtime. These checks are generated and inserted in the resulting JavaScript code automatically, which helps to identify possible issues during the development of a project the native TypeScript compiler cannot detect.
