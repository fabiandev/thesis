%TODO: sources? references?
\chapter{Introduction}
\label{cha:introduction}

JavaScript is a popular programming language on the client- and server-side. It has evolved considerably in recent years and the latest specification called ECMAScript 2015---also known as ES6---which, among other things, introduced classes to native JavaScript, was a major step for developers. Even though JavaScript's dynamic type system can be of advantage in a lot of scenarios, it adds the risk of introducing errors. The language tries to perform type conversions in situations where values are not compatible (see sec.~\ref{sec:type-conversion} and~\ref{sec:value-comparison}), which can lead to unexpected behavior. Therefore TypeScript came up with a superset of JavaScript (see sec.~\ref{sec:typescript}), giving developers the ability to optionally add type annotations to their projects, resulting in increased code readability, scalability and maintainability. TypeScript's static compile time type checks further can detect a multitude of conditions, that may cause issues in the target code at runtime. The compiler also adds support for the latest JavaScript features and proposals~\cites{TypeScriptHandbook:CompilerOptions, TypeScriptWebsite}, which enables the usage of future language characteristics that are not yet supported.

%, with increased code readability, and to keep projects maintainable and better structured. It also adds various other language features, such as protected or private class members.

\section{Problem Definition}
\label{sec:problem-definition}

Although type compatibility is checked during compile time, it is not available in the compiled JavaScript code. The removal of types is intended by Microsoft and is defined in the design goals of the language~\cite{TypeScriptWiki:DesignGoals}. A number of issues have been filed on Microsoft's GitHub repository, requesting the ability to automatically generate runtime type checks~\cites{TypeScriptIssue:RuntimeTypeChecking, TypeScriptIssue:RuntimeTypeChecks, TypeScriptIssue:EmitTypeArguments}, which were rejected due to being out-of-scope the language's goals~\cites{TypeScriptIssue:RuntimeTypeChecking:Comment:OutOfScope, TypeScriptIssue:EmitTypeArguments:Comment:OutOfScope}. Therefore extensive type checks have to be performed manually for situations the compiler cannot detect them (see sec.~\ref{sec:undetectable-errors}), such as HTTP requests or untyped third party code. This results in increased development effort and greater susceptibility to errors (see sec.~\ref{sec:runtime-type-checks}).

\section{Solution Approach}
\label{sec:solution-approach}

Given the fact that suitable type information is available for most situations---either through type annotations or type inference---suggests, that generating runtime type checks based on the existing data at compile time is technically possible. The information, that is usually removed by the TypeScript compiler, should be reflected for the target code, to obtain it for type compatibility checks during the running of the program. These checks should be generated and inserted in the resulting JavaScript code automatically, which should help to identify possible issues during the development of a project, the TypeScript compiler is not able to detect. In order to achieve a desirable result, situations where verifications need to take place, have to be identified carefully. Also the footprint of code added to a project, as well as the performance impact on the program being executed, should be as small as possible. While the purpose of this project is to be used in the phase of development, employing its technique in a production environment should remain in focus as well.

\section{Thesis Structure}
\label{sec:thesis-structure}

While this chapter's intention was to give an exposure to the contents of this thesis, in chapter~\ref{cha:technical-foundation} the technical foundation, including definition of terminology and the introduction of the programming language JavaScript, as well as various concepts, implied for the remaining chapters, will be introduced. In chaper~\ref{cha:state-of-the-art} the superset TypeScript will be explored and compared to a similar project, and an overview of the current state on automated runtime type checks for JavaScript will be acquired. After the rudiments of the topic have been handled and the state of the art in the field of runtime type checks for JavaScript has been examined, the theoretical approach for the thesis project will be elaborated in chapter~\ref{cha:theoretical-approach}, following its implementation in chapter~\ref{cha:implementation}. Eventually, the result will be evaluated in chapter~\ref{cha:evaluation}, before summarizing the outcome of the thesis, as well as giving an outlook into the future in chapter~\ref{cha:summary_outlook}.
