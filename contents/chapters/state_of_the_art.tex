\chapter{State of the Art}
\label{cha:state-of-the-art}

The following chapter will provide an overview of available technologies and techniques to introduce a static type system---as defined in Section~\ref{sec:type-systems}---to JavaScript. The main focus will be on TypeScript, but also other supersets will be explored. Also the current state of type checks in JavaScript will be highlighted and existing projects, to automatically generate runtime type checks, will be explored.

\section{JavaScript Supersets}
\label{sec:supersets}

\citeauthor{Term:Superset} defines a \emph{superset} as 
\begin{quote}
   [a] set containing all elements of a smaller set. If $B$ is a subset of $A$, then $A$ is a superset of $B$ [...]~\cite{Term:Superset}.
\end{quote}
This means that every program that is valid in JavaScript is also legal in a JavaScript superset, where the purpose of such a superset is to add features to the original language. As the source written in the supersets syntax will be compiled to JavaScript, any additional functionality needs to be representable as JavaScript code.

\subsection{TypeScript}
\label{sec:typescript}

TypeScript was created by Anders Hejlsberg---the designer of C\texttt{\#}---at Microsoft~\cite[p.~10]{MasteringTypeScript:Rozentals:2017} and was released in 2012 under the Apache 2.0 open source license~\cite[p.~xix]{ProTypeScript:Fenton:2014}. The most important aspect of TypeScript is, that it includes a compilation step, where static type checking is performed~\cite[p.~11]{MasteringTypeScript:Rozentals:2017}. Type annotations are optional and the compiler will infer type information where possible~\cite[p.~10]{TypeScriptBook:Syed:2017}. TypeScript also introduces concepts known from other programming languages, such as interfaces and enumerations (i.e., enums). Not only it is possible to develop a program in the TypeScript syntax, but also to add type annotations to existing JavaScript projects~\cite[p.~13]{MasteringTypeScript:Rozentals:2017}. The most significant particularities and features will be explored in this section.

\subsubsection{Basic Types}
\label{sec:ts-basic-types}

Like many major languages, TypeScript defines some basic types, that overlap with JavaScript's types, listed in Section~\ref{sec:value-types}, while also introducing new ones~\cite{TypeScriptHandbook:BasicTypes}:
\begin{itemize}
  \item \emph{Tuple} is a special kind of an array, only allowing a fixed number of elements.
  \item \emph{Enum} may already be known from other programming languages, like Java, and is useful to define a set of numeric values.
  \item \emph{Any} results in type checking not being performed by the compiler. This is useful when using TypeScript alongside third-party-libraries where no type definitions are available. It also allows e.g. access to any property on an object, whether it does exist or not.
  \item \emph{Void} is the counterpart to Any. Again, it is used in other languages, for example to annotate functions that do not return a value. In TypeScript also variables may be typed as \texttt{void}, meaning that only \texttt{undefined} or \texttt{null} will be accepted as value.
  \item \emph{Never} for instance is useful for functions that always throw an error or result in an infinite loop, as no value will ever come back~\cite{TypeScriptHandbook:BasicTypes}.
\end{itemize}
While other types, such as \emph{Function} or more advanced structures are also available, the ones listed in combination with those already defined in JavaScript (i.e., Undefined, Null, Boolean, String, Symbol, Number, and Object) are the most important ones to get started.

\subsubsection{Type Inference}
\label{sec:ts-type-inference}

As already mentioned, TypeScript infers the type if possible. The snippet below shows a simple variable declaration in JavaScript (or TypeScript), where the TypeScript compiler can automatically infer the type Number from the declaration.
\begin{JsCode}[numbers=none]
let num = 1;
\end{JsCode}
Therefore it won't allow any subsequent assignment to \texttt{num} not being a number. For example the reassignment \texttt{num = "foo"} would result in the following compiler diagnostic:
\begin{JsCode}[numbers=none]
Type '"foo"' is not assignable to type 'number'.
\end{JsCode}
The term \emph{diagnostic} over \emph{error} is used here, because by default the compiler won't stop in such cases and will do its best to emit the final JavaScript code~\cite[p.~12]{TypeScriptBook:Syed:2017}:
\begin{JsCode}[numbers=none]
let num = 1;
num = "foo";
\end{JsCode}
The code above shows the resulting JavaScript program, even though the compiler detected a type error.

\subsubsection{Type Annotations}
\label{sec:ts-type-annotations}

While type inference can be useful in some situations, others require types to be set explicitly, as shown below:
\begin{JsCode}[numbers=none]
let num: number;
num = 1;
\end{JsCode}
The variable \texttt{num} was declared, but not initialized, requiring  a type annotation, in order to be treated as a number by TypeScript. Omitting the explicit type information, the compiler would infer the Any type, allowing arbitrary assignments to the variable.

\subsubsection{Type Assertions}
\label{sec:ts-type-assertions}

Type assertions are a way to provide TypeScript with type information, which is not available to the compiler. They are 
\begin{quote}
  [...] like [...] type [casts] in other languages, but [perform] no special checking or restructuring of data~\cite{TypeScriptHandbook:BasicTypes}.
\end{quote}
It is the developer that has to take care of performing sufficient checks when using a type assertion. Because of the possibility to use any existing JavaScript library with TypeScript, situations where the compiler does not have any type information of the external package available may occur. Type assertion can be a solution to prevent compile time type errors in such cases:
\begin{JsCode}[numbers=none]
import RandomName from "random-name";
let name: string = (RandomName as any).getName();
\end{JsCode}
In the sample from above, the default export from the library \texttt{random-name} was imported as \texttt{RandomName}. This package is neither written in TypeScript nor does it have type definitions available. However, the library has a callable \texttt{getName} property, returning a string. As the compiler is not aware of the package's properties and their return types, it is necessary to tell it which type to assert. \texttt{RandomName} was casted to \texttt{any}, allowing property access independently of their existence. Again, because of type assertions (or type casts) not performing any special checking, the solution from above may lead to errors if the author of the package decides to change its application programming interface (i.e., API). Therefore manually checking for if \texttt{RandomName} has a callable \texttt{getName} property and is actually returning a string may be recommended here.
As an alternative to the type casting syntax with the \texttt{as} keyword, the following may be used:
\begin{JsCode}[numbers=none]
let name: string = (<any>RandomName).getName();
\end{JsCode}
However, this \emph{angle-bracket} syntax is not supported when using TypeScript with \emph{JSX\footnote{``JSX is an embeddable XML-like syntax [...] meant to be transformed into valid JavaScript [which] came to popularity with the React framework, but has since seen other applications as well.''~\cite{TypeScriptHandbook:JSX}}}~\cite{TypeScriptHandbook:BasicTypes}, making the \texttt{as} syntax preferable.

\subsubsection{Ambient Type Declarations}
\label{sec:ts-ambient-type-declarations}

In TypeScript either existing structures---such as classes and basic types---can be used as type annotation, or they can be defined via interfaces or type aliases. The latter are not part of the code after compilation, while e.g.\ classes or enums remain in the JavaScript code. Anyway, it is also possible to declare, among others, a class or variable as ambient in TypeScript. This may be useful when consuming a third party package, that was not written in TypeScript and there are no type definitions for it available. In the previous section type casting was used to circumvent this issue. While this is a possibility, it may not be suitable, if the library is used frequently in a project. Prepending e.g.\ a variable, class, namespace or enum with the \texttt{declare} keyword, makes them ambient and will be treated by the compiler as if they were part of the code:
\begin{JsCode}[numbers=none]
declare const RandomName: any;
\end{JsCode}
This results in TypeScript always treating the variable \texttt{RandomName} as having type \texttt{any}. Alternatively the imported package may be described in more detail:
\begin{JsCode}[numbers=none]
declare const RandomName: {
  getName: () => string;
};
\end{JsCode}
From now on, TypeScript will know, that the import has a callable property \texttt{getName} that returns a string.

\subsubsection{Structural Types}
\label{sec:ts-structural-types}

Types in TypeScript are structural~\cite[p.~11]{TypeScriptBook:Syed:2017}, meaning that the type checker looks at members of an object, to ensure type compatibility, while other major languages, such as C\texttt{\#} or Java, use nominal type systems~\cite{TypeScriptHandbook:TypeCompatibility}. Program~\ref{prog:structural-typing} gives an example that would fail in a nominally typed language, but is possible in TypeScript.

\begin{program}
\caption{An instance of \texttt{Person} can be assigned to a variable with type \texttt{Named} on line 10, because of TypeScript's structural type system. In languages with a nominal type system the class \texttt{Person} would need to implement the interface \texttt{Named} in their corresponding syntax, for this example to be valid~\cite{TypeScriptHandbook:TypeCompatibility}.}
\label{prog:structural-typing}
\begin{JsCode}
interface Named {
    name: string;
}

class Person {
    name: string;
}

let p: Named;
p = new Person();
\end{JsCode}
\end{program}

\subsubsection{Classes}
\label{sec:ts-classes}

TypeScript not only enables static type checking for JavaScript applications, but also includes additional language features. While EcmaScript 2015 introduced classes, with TypeScript it is possible to also define them as abstract and to add visibility modifiers and interface implementations to them as shown below.
\begin{JsCode}[numbers=none]
class Person implements Human {
  public name: string;
  private age: number;
}
\end{JsCode}
The keywords \texttt{public}, \texttt{protected} and \texttt{private} may be used for class members and methods.
Also it is possible to define members and provide default values outside of the constructor, as well as to mark properties as \texttt{readonly}, preventing any reassignment.
\begin{JsCode}[numbers=none]
class Person implements Human {
  public readonly id = uid();
}
\end{JsCode}
Anyway, it is important to note, that the modifiers described, as well as implemented interfaces, are only relevant during compile time. After the final JavaScript code has been emitted, this information is stripped out and cannot be used in the running program, as shown in the compiled class, using plain JavaScript syntax, below.
\begin{JsCode}[numbers=none]
class Person {
  constructor() {
    this.id = uid();
  }
}
\end{JsCode}
This means that it is technically possible, to assign an arbitrary value to \texttt{id} of a \texttt{Person} instance during runtime.

\subsubsection{Enums}
\label{sec:ts-enums}

As already mentioned earlier, an enumeration is beneficial for defining a set of numeric values. The TypeScript compiler will take any enum declaration and transform it into runnable JavaScript code. The following enum in TypeScript syntax
\begin{JsCode}[numbers=none]
enum HairColor {
  Black, Blond, Brown, Red, Other
}
\end{JsCode}
results in the JavaScript code below, which creates an object containing the values defined in the enumeration.
\begin{JsCode}[numbers=none]
var HairColor;
(function (HairColor) {
    HairColor[HairColor["Black"] = 0] = "Black";
    HairColor[HairColor["Blond"] = 1] = "Blond";
    // ...
})(HairColor || (HairColor = {}));
\end{JsCode}
The enum can now be accessed during runtime to obtain the corresponding numeric value. Also it is possible, to reveal the name of a numeric value from the enumeration:
\begin{JsCode}[numbers=none]
HairColor.Black // 0
HairColor[0] // "Black"
\end{JsCode}
However, if the enumeration is declared as constant, the compiler will look up the numeric value and will insert it directly into the source code, before entirely removing its definition~\cite{TypeScriptHandbook:Enums}.

\subsubsection{Namespaces}
\label{sec:ts-namespaces}

In TypeScript, namespaces provide a possibility to encapsulate code. They were previously referred to as \emph{internal modules}, but have since been renamed to avoid confusion with native \emph{modules} of the EcmaScript standard, previously denoted as \emph{external modules} in TypeScript~\cite{TypeScriptHandbook:Namespaces}. Code within a namespace only expose explicitly exported parts.
\begin{JsCode}[numbers=none]
namespace Capsule {
  let foo = "Hello from Capsule!";
  
  export function bar() {
    return foo;
  }
}
\end{JsCode}
Accessing \texttt{foo} of the namespace \texttt{Capsule} would result in \texttt{undefined}, whereas calling \texttt{bar} would return the value of \texttt{foo}. If taking a look at the JavaScript code, generated from the namespace above, this behavior is made clear.
\begin{JsCode}[numbers=none]
var Capsule;
(function (Capsule) {
    var foo = "Hello from Capsule!";
    function bar() {
        return foo;
    }
    Capsule.bar = bar;
})(Capsule || (Capsule = {}));
\end{JsCode}
A variable with the name of the namespace will be declared and an empty object will be assigned to it. Only the namespace's exported parts will be added to this object to be exposed, while all other values 
remain exclusively accessible from within the self executing function itself.

\subsubsection{Parameter Default Values}
\label{sec:ts-parameter-default-values}

Another useful feature is the possibility to define default values for parameters in TypeScript. This gives developers the ability to avoid parameters being \texttt{undefined} if not passed, and can be useful in various other scenarios.
\begin{JsCode}[numbers=none]
function log(message: string, logger = console) {
  logger.log(message);
}
\end{JsCode}
The example above shows a simple log function which writes a string to the console, when omitting the second parameter. If another log mechanism is desired, it is possible to pass a different logger to this method, which also implements the Console interface, or at least shares its properties.
The same result can be achieved with plain JavaScript as well, as shown in the compiled code below:
\begin{JsCode}[numbers=none]
function log(message, logger) {
    if (logger === void 0) { logger = console; }
    logger.log(message);
}
\end{JsCode}
If the parameter \texttt{logger} equals \texttt{void 0}, which is identical to \texttt{undefined}\footnote{The void operator can be used to retrieve the value \texttt{undefined} by calling \texttt{void(0)}, which is equivalent to \texttt{void 0}~\cite{void:MDN:2017}.}, the global variable \texttt{console} will be assigned to it. Otherwise the parameter passed to the function \texttt{log} will be used as is.

% enums, namespaces, decorators, (abstract) class (visibility/inheritance/implements), default value for parameter, readonly

\subsubsection{Future JavaScript}
\label{sec:ts-future-javascript}

While the TypeScript compiler can target different JavaScript versions, such as ES3, ES5 and ES2015, it also supports future ECMAScript proposals, like decorators and asynchronous functions~\cites{TypeScriptHandbook:CompilerOptions, TypeScriptWebsite}. This allows for the usage of features, that are not yet implemented by browsers, which is achieved by changing parts of the source or including additional code that mimics the behavior of a certain feature and delivers the same result. The code below uses a pattern, referred to as \emph{destructuring assignment}~\cite{DestructuringAssignment:Mozilla:2015}, to assign the values \texttt{1} and \texttt{2} to the variables \texttt{foo} and \texttt{bar} respectively:
\begin{JsCode}[numbers=none]
let [ foo, bar ] = [1, 2];  
\end{JsCode}
While this line would remain unchanged when targeting the ES2015 standard or higher, where the \emph{array binding pattern} is already specified~\cite[p.~198]{ES6Spec:Ecma:2015}, the outcome is different for ES5 and below:
\begin{JsCode}[numbers=none]
var _a = [1, 2], foo = _a[0], bar = _a[1];
\end{JsCode}
As the pattern is not specified in the fifth edition of ECMAScript~\cite{ES5Spec:Ecma:2015}, the compiler exchanges the line with an alternative implementation, as shown in the snippet above.
 
\subsection{Flow}
\label{sec:flow}

\emph{Flow\footnote{\url{https://flow.org}}} is another open source static type checker for JavaScript, developed by \emph{Facebook\footnote{\url{https://code.facebook.com}}}~\cite{FacebookCode:Flow}. The most noticeable difference to TypeScript is the lack of an extensive compiler provided by the project itself. Instead, Flow relies on \emph{Babel\footnote{\url{https://babeljs.io}}}, which is a compiler for JavaScript~\cite{BabelWebsite}, with
\begin{quote}
  [...] support for Flow [that] will take [...] Flow code and strip out any type annotations.~\cite{FlowDocs:Install}
\end{quote}
Alternatively the library \emph{flow-remove-types\footnote{\url{https://github.com/flowtype/flow-remove-types}}} can be used~\cite{FlowDocs:Install}. Another difference between the two JavaScript supersets are their design goals. While TypeScript goal is not to ``[a]pply a sound or "provably correct" type system [but to] strike a balance between correctness and productivity.~\cite{TypeScriptWiki:DesignGoals}'', Flow's type system ``tries to be as sound and complete as possible~\cite{FlowDocs:TypesAndExpressions}''. The syntax of Flow itself is mostly identical to the one of TypeScript~\cite{FlowDocs:TypesAnnotations}. Brzóska sums up the differences between the two languages, as shown in Table~\ref{tab:typescript-flow}.
%\begin{table}
%\caption{Differences between TypeScript and Flow.~\cite{TypeScriptVsFlow}}
%\label{tab:typescript-flow}
%\centering
%\def\rr{\rightskip=0pt plus1em \spaceskip=.3333em \xspaceskip=.5em\relax}
%\setlength{\tabcolsep}{1ex}
%\def\arraystretch{1.20}
%\setlength{\tabcolsep}{1ex}
%\small
%\begin{tabular}{|c|p{0.35\textwidth}|p{0.35\textwidth}|}
%\hline
%   \multicolumn{1}{|c}{} &
%   \multicolumn{1}{|c}{\emph{TypeScript}} &
%   \multicolumn{1}{|c|}{\emph{Flow}} \\
%\hline\hline
%    Design Goal &
%   {\rr balance between correctness and productivity} &
%   {\rr enforce type soundness and safety} \\
%\hline
%\end{tabular}
%\end{table}
\begin{table}
\caption{Differences between TypeScript and Flow.~\cite{TypeScriptVsFlow}}
\label{tab:typescript-flow}
\centering
\setlength{\tabcolsep}{5mm}
\def\arraystretch{1.25}
\small
\begin{tabular}{|r||c|c|c|}
    \hline
    & \emph{TypeScript} & \emph{Flow} \\
    \hline
    \hline
    Design Goal &
    \makecell{correctness and productivity} &
    \makecell{soundness and safety} \\
    \hline
    IDE Integrations &
    top-notch &
    sketchy \\
    \hline
    Autocomplete &
    yes &
    unreliable \\
    \hline
    Speed &
    stable &
    degrades \\
    \hline
    Generic Definitions &
    yes &
    yes \\
    \hline
    Generic Calls &
    yes &
    no \\
    \hline
    Library Typings &
    many &
    few \\
    \hline
  \end{tabular}
\end{table}

\subsection{Others}
\label{sec:other-supersets}

Apart from TypeScript and Flow, a multitude of languages exist that compile to JavaScript, some of them being supersets, with different purposes. The \emph{CoffeeScript\footnote{\url{https://github.com/jashkenas/coffeescript}}} project collected an extensive list, containing the following maintained languages they refer to as supersets~\cite{LanguagesToJavaScript}:
\begin{itemize}
  \item \textbf{JavaScript\texttt{++}:} Supports classes, type checking, and other features.
  \item \textbf{Objective-J:} Same relationship to JavaScript as \emph{Objective-C} to \emph{C}.
  %\item \textbf{Six:} Supports features from the 6th edition of ECMAScript through a transpiler
  %\item \textbf{Latte JS:} Similar to CoffeeScript with JavaScript syntax
  \item \textbf{JSX:} Adds XML-like syntax to represent HTML elements.
  \item \textbf{oj:} Objective-C inspired superset with an experimental type checker.
\end{itemize}
The collection also contains languages like \emph{Scala.js\footnote{\url{http://www.scala-js.org}}}, which compiles \emph{Scala\footnote{\url{http://scala-lang.org}}} code to JavaScript, or \emph{Opal\footnote{\url{http://opalrb.org}}}, a \emph{Ruby\footnote{\url{https://www.ruby-lang.org}}} to JavaScript compiler~\cite{LanguagesToJavaScript}.

\section{Runtime Type Checks}
\label{sec:runtime-type-checks}

%The situations discussed in the previous sections of this chapter raised the concern of negligently or unknowingly opting out of static type checks for specific parts of the code or of providing insufficient or wrong type declaration, which may cause the compiler to miss incompatible types. These situation may be discoverable at runtime, if runtime type checks would be employed alongside compile time checks.

%As already mentioned in section~\ref{sec:typescript},
Type annotations are stripped out for the compiled JavaScript program in TypeScript and no additional code is introduced to add checks at runtime. The removal of types is intended and is defined in the design non-goals of the language~\cite{TypeScriptWiki:DesignGoals}:
\begin{quote}
  [Do not] add or rely on runtime type information in programs,
  or emit different code based on the results of the type system.
  Instead, encourage programming patterns that do not require runtime metadata.
\end{quote}
Anyway, runtime type information and checks can be useful in several situations. 
For example they can give more accurate error messages during development and can 
flag issues, that are not observable during compile time.
There are proposals, to expose type information to the runtime and to add runtime type
checks, in the TypeScript community~\cites{TypeScriptIssue:RuntimeTypeChecking, TypeScriptIssue:RuntimeTypeChecks, TypeScriptIssue:EmitTypeArguments}. Regardless of the demand, these features won't be added, as they are out of scope
for TypeScript~\cite{TypeScriptIssue:RuntimeTypeChecking:Comment:OutOfScope, TypeScriptIssue:EmitTypeArguments:Comment:OutOfScope}.
Currently manually added type checks are required to identify and easily trace errors during development. Program~\ref{prog:js-without-typechecks} shows a simple JavaScript function, which only has three lines of code, while Program~\ref{prog:js-with-typechecks} shows a function with the same outcome, but added type 
checks---also in native JavaScript---that now requires 13 lines of code.
\begin{program}
\caption{A JavaScript function without type checks.}
\label{prog:js-without-typechecks}
\begin{JsCode}
function sum(arr) {
  return arr.reduce((a, b) =>  a + b);
}
\end{JsCode}
\end{program}
\begin{program}
\caption{The JavaScript function from Program~\ref{prog:js-without-typechecks} with type checks.}
\label{prog:js-with-typechecks}
\begin{JsCode}
function sum(arr) {
  if (!Array.isArray(arr)) {
    throw new TypeError("array expected");
  }
  
  return arr.reduce((a, b) => {
    if (typeof b !== "number") {
      throw new TypeError("number expected");
    }
    
    return a + b;
  });
}
\end{JsCode}
\end{program}
While these examples outline the verification of primitive types, like a number or an array, inspecting an object is more complex. Instances may be checked with the \texttt{instanceof} operator, which ``[...] tests whether an object in its prototype chain has the prototype property of a constructor~\cite{instanceof:MDN:2017}'', but interfaces or type alias are removed by the TypeScript compiler, therefore such a verification is not possible for such cases. To get around this issue, \citeauthor{MasteringTypeScript:Rozentals:2015} describes three different techniques to employ type checks for the runtime engine:
\begin{itemize}
  \item \textbf{Reflection:} The prototype of a JavaScript object holds 
some information about the object which can be accessed. It might, for instance, contain the name of the constructor function, used to create the object. Limitations apply, as various
information is only available from ECMAScript 5.1, or may not be available at all~\cite[pp.~98--100]{MasteringTypeScript:Rozentals:2015}. Also the name of a constructor is not always suitable to categorize an object as a type, as the same name may also be used for a different constructor function or an anonymous function has been used. Simply obtaining the name is also not sufficient to check for implemented interfaces or type aliases, as they are compiled away by TypeScript.
  \item \textbf{Checking an object for a property:} An object could be considered 
as being of a type, if specified properties exist on it~\cites[pp.~101--102]{MasteringTypeScript:Rozentals:2015}[pp.~18--20]{ProJavaScriptDesignPatterns:HarmesDiaz:2008}. For example every \textit{Person} object must implement a \textit{getName} method. By verifying if a function with this name is available on the object, it may be considered as the given type. This gets closer to TypeScript's structural types (see Sec.~\ref{sec:ts-structural-types}).
  \item \textbf{Interface checking with generics:} For every interface there is a 
class, which holds its property names, to identify an object as 
having a specific type~\cites[pp.~102--105]{MasteringTypeScript:Rozentals:2015}[pp.~17--19]{ProJavaScriptDesignPatterns:HarmesDiaz:2008}. This solution is similar to the previous one, but it introduces a pattern which is more readable and maintainable.
\end{itemize}
Another mechanism is to use \textit{decorators\footnote{\url{https://tc39.github.io/proposal-decorators}}},
a JavaScript language feature proposal, which is currently at stage two~\cite{DecoratorsProposalRepo}, meaning that it is still a draft and not yet in the specification~\cite{EcmaScriptProposalProcess}. They can, however, already be used with TypeScript or tools like \emph{Babel\footnote{\url{https://babeljs.io}}}~\cite{TypeScriptHandbook:Decorators, Babel:Plugins:Decorators}.
The solution used in~\cite{DecoratorTypeChecks}, that makes use of decorators, again requires
to add them to the source code manually. Furthermore only primitive types and instances can be checked automatically. Structural type checks---e.g.\ for custom objects or interfaces---have
to be provided by the developer.

As runtime type checks are of importance for an application to be robust, the operators \texttt{typeof} and \texttt{instanceof} are often used to verify a value's type, which according to \citeauthor{DynamicTypeChecks:Rauschmayer:2017} is ``[...] less than ideal, because [it requires] to keep the difference between primitive values and objects in mind~\cite{DynamicTypeChecks:Rauschmayer:2017}''. Program~\ref{prog:runtime-checks-hasInstance} shows a technique to enable \texttt{instanceof} checks also for primitive values, such as strings. He further refers to using a library for checking types at runtime, and outlines two ECMAScript proposals that are related to runtime validations~\cite{DynamicTypeChecks:Rauschmayer:2017}, which are shown in Program~\ref{prog:runtime-checks-pattern-match} and~\ref{prog:runtime-checks-builtin}.

\begin{program}
\caption{The following code overwrites the default \texttt{instanceof} behavior for the given class.~\cites{DynamicTypeChecks:Rauschmayer:2017, DynamicTypeChecks:hasInstance}}
\label{prog:runtime-checks-hasInstance}
\begin{JsCode}
class PrimitiveNumber {
  static [Symbol.hasInstance](x) {
    return typeof x === "number";
  }
}

123 instanceof PrimitiveNumber; // true
\end{JsCode}
\end{program}

\begin{program}
\caption{The ECMAScript proposal for pattern matching would add a sophisticated validation pattern in JavaScript.~\cites{DynamicTypeChecks:Rauschmayer:2017, PatternMatchingProposalRepo}}
\label{prog:runtime-checks-pattern-match}
\begin{JsCode}
match (obj) {
  { x }: /* match an object with x */,
  { x, ... y }: /* match an object with x, stuff any remaining properties in y */,
  { x: [] }: /* match an object with an x property that is an empty array */,
  { x: 0, y: 0 }: /* match an object with x and y properties of 0 */
}
\end{JsCode}
\end{program}

\begin{program}
\caption{The code below shows an ECMAScript proposal for \texttt{Builtin.is}, which ``[...] checks if [two values] refer to the same builtin constructor~\cite{DynamicTypeChecks:Rauschmayer:2017}'' and \texttt{Builtin.of}, ``[...] an extension typeof that works for both primitive values and built-in classes~\cite{DynamicTypeChecks:Rauschmayer:2017}''.~\cites{DynamicTypeChecks:Rauschmayer:2017, BuiltinProposalRepo}}
\label{prog:runtime-checks-builtin}
\begin{JsCode}
Builtin.is(Date, Date); // true

class MyArray extends Array { }
Builtin.typeOf(new MyArray()); // "Array"
\end{JsCode}
\end{program}

% All of these approaches have in common, that it requires developers to manually add checks, resulting in increased development effort.

\section{Generated Runtime Type Checks}
\label{sec:existing-projects}

Currently there are no libraries available to automatically generate runtime type checks from TypeScript code, and validations have to be included manually, as described in the previous section. However, there are libraries that aim to provide a runtime type system, which are explored in Section~\ref{sec:runtime-type-system}. While those packages are supportive in describing and validating data structures in JavaScript, few projects concentrate on automatically generating them. As it may only be feasible to create checks based on information provided by a static type system, or supportive data such as type annotations, the \emph{Babel} plugins \emph{babel-plugin-tcomb\footnote{\url{https://github.com/gcanti/babel-plugin-tcomb}}} and \emph{babel-plugin-flow-runtime\footnote{\url{https://github.com/codemix/flow-runtime/tree/master/packages/babel-plugin-flow-runtime}}} can generate runtime validations for Flow syntax~\cites{RuntimeTypeSystem:tcomb:babel, RuntimeTypeSystem:flow-runtime:babel}. However, a future release of Babel will support TypeScript syntax~\cite{Babel:TypeScript}, which would make it possible to adapt the plugins, to also transform TypeScript code.
