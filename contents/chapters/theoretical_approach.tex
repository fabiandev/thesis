\chapter{Theoretical Approach}
\label{cha:theoretical-approach}

After defining the terminology for this thesis, as well as giving an overview of available technologies and projects, the theoretical approach for generating runtime type checks from TypeScript type annotations for runnable JavaScript code is described in this chapter. The project of this thesis will further be referred to as \emph{ts-runtime} (i.e., \emph{typescript-runtime}) or \emph{tsr}.

\section{Undetectable Errors}
\label{sec:undetectable-errors}
%
%This section is dedicated to point out, why runtime type checks are of importance. Also current approaches and solutions are discovered, to get an overview of the techniques available and steps required to make a TypeScript project more resistant to failure.
%
%\subsection{Undetectable Errors}
%\label{sec:undetectable-errors}

There are various situations where the static type analysis of TypeScript cannot detect conditions that may lead to errors during runtime. Either a project is written in TypeScript, and the compiler can infer the type information needed, or type definition files are provided for untyped JavaScript libraries. In both cases it is possible to introduce errors, which may cause the type checker to make wrong assumptions about type compatibility. Also particular programming techniques can result in errors not already being trapped during compilation.

\subsection{Compiler Analysis Circumvention}

In TypeScript it is possible to perform a special kind of type cast, called \emph{type assertion}, as described in Section~\ref{sec:ts-type-assertions}. While the compiler will trigger an error, when trying to assert an incompatible type---e.g., \text{asserting} a string as a number---there exists a special case to bypass static type checks for a given variable or value entirely. If e.g., a variable is annotated or asserted with the \emph{any} type, type checking and type inference will be disabled this part of the code. The TypeScript documentation describes this type as 
\begin{quote}
   [...] a powerful way to work with existing JavaScript, allowing you to gradually opt-in and opt-out of type-checking during compilation.~\cite{TypeScriptHandbook:BasicTypes}
\end{quote}
It seems legitimate to use \emph{any} alongside third-party libraries or in situations where the flexibility of JavaScript's loose typing (see Sec.~\ref{sec:untyped-loosely-typed}) is required. However, opening the possibility to opt-out of type checks can have a negative impact for projects depending on libraries where this technique is misused. The following code outlines a situation where compilation passes, but an error is thrown at runtime:
\begin{JsCode}[numbers=none]
let foo: any = "bar";
foo.getNumber();
\end{JsCode}
Because of type checks being disabled for variable \texttt{foo}, access to the not existing property \texttt{getNumber} won't be detected by the compiler. Even if the identifier was annotated correctly, or its type could be inferred by omitting a type annotation, it is possible to get around type checks:
\begin{JsCode}[numbers=none]
let foo: string = "bar";
(foo as any).getNumber();
\end{JsCode}
In both cases the JavaScript runtime engine will throw a \emph{TypeError} exception, stating that \emph{foo.getNumber is not a function}. The examples from above highlight the potentiality of creating conditions where a detectable mistake remains undiscovered by the compiler, which could cause a running program to be interrupted.

\subsection{Polymorphism}

Polymorphism can lead to errors at runtime in combination with type assertions. While the TypeScript compiler does check type compatibility in general, it allows to assert identifiers as types, which could be assigned to it. To give an example of such a situation the following code is given:
\begin{JsCode}[numbers=none]
class Animal { }

class Cat extends Animal {
  miow() {
    return "Miow";
  }
}

class Dog extends Animal {
  woof() {
    return "Woof";
  }
}
\end{JsCode}
As a next step an instance of \texttt{Dog} is created and assigned to a variable, which is typed as \texttt{Animal}, as shown below:
\begin{JsCode}[numbers=none]
let dog: Animal = new Dog();
\end{JsCode}
In order to call the method \texttt{woof} on the \texttt{Dog} instance, it needs to be asserted as follows:
\begin{JsCode}[numbers=none]
(dog as Dog).woof();
\end{JsCode}
The TypeScript compiler does not raise any concern, as type \texttt{Dog} is assignable to \texttt{Animal} and therefore is allowing the cast. Subsequently, also the following type assertion passes without any compiler errors:
\begin{JsCode}[numbers=none]
(dog as Cat).miow();
\end{JsCode}
While static type checks are successful, the compiled JavaScript code will fail at runtime. As no method \texttt{miow} exists on the \texttt{dog} object, a \emph{TypeError} exception with the message \emph{dog.miow is not a function} will be thrown at runtime.

\subsection{Untyped JavaScript Libraries}

If a JavaScript project is written in native syntax, TypeScript cannot infer the type information needed to perform sufficient static type analysis. In this cases type declarations may be provided manually, as discussed in the following section. \emph{DefinitelyTyped\footnote{http://definitelytyped.org}} provides a collection of such type definitions for JavaScript libraries~\cite{DefinitelyTyped}. Anyway, not all packages have definitions available, especially small projects. A practice often used in such a situation is to declare the package name, to stop complaints from the TypeScript compiler about not finding the import:
\begin{JsCode}[numbers=none]
declare module "MyModule"
\end{JsCode}
After declaring the module it effectively has the \emph{any} type applied to it. Therefore, as already mentioned in the previous section, it is possible to access any property on the imported module, regardless of whether it exists or not. Also changes to the package's API won't be noticed, and a project depending on the module may break after updating its dependencies.

\subsection{Type Declaration Mistakes}

Libraries written in TypeScript are usually published as compiled JavaScript code alongside type declaration files with the extension \emph{d.ts}. These files include all type information that was removed for the runnable JavaScript code. The compiler can parse the definitions and can statically check the correct usage of the library, when imported in another project. If the definitions are generated during compilation, they can be considered relatively safe to use, unless the \emph{any} type is misused. If, however, declaration files are created manually for a JavaScript library, which is not written in TypeScript, there is a chance that they contain mistakes or that they are not be up-to-date with the implementation. A JavaScript file \emph{foo.js} may contain the following code:
\begin{JsCode}[numbers=none]
class Foo {
  getName(): string {
    return "Foo";
  }
}
\end{JsCode}
Its corresponding declaration file \emph{foo.d.ts} provides the declaration as shown below:
\begin{JsCode}[numbers=none]
declare class Foo {
  getNumber(): number;
}
\end{JsCode}
The file containing the type declaration for class \texttt{Foo} does not reflect the actual implementation. The method \texttt{getNumber}, as suggested by the type definition, does not exist on the class. Code completion in an IDE (i.e., integrated development environment) may suggest to use this method, which can lead to a runtime exception. Also the static type checker would complain if attempting to use the implemented method \texttt{getName}, as it has no knowledge of its existence on the object.

\subsection{Erroneous API Responses}

When making use of an external API, the response should be of a given structure, which should be known to the consuming program. An interface may be created, to describe the receiving object and to statically check its usage. However, if the format of the response changes unexpectedly, it is not possible to reveal this change at compile time, as the type checker relies on the interface provided. Runtime checks have to be added manually to ensure that the response conforms to the expected structure. Otherwise an error may be raised during program execution, or its behavior may be different than expected.

%\subsection{Source Code Manipulation}
%\label{sec:source-code-manipulation}

%\section{Method Principle}
%\label{sec:method-principle}

\section{Desired Result}
\label{sec:desired-result}

The situations discussed in the previous section of this chapter raised the concern of negligently or unknowingly opting out of static type checks for specific parts of the code or of providing insufficient or wrong type declaration, which may cause the compiler to miss incompatible types. These situation may be discoverable at runtime, if runtime type checks would be employed alongside compile time checks. Information about types are already available in a TypeScript development environment. Either it is provided through explicit type annotations, or the compiler tries to infer the type, which leads to the assumption, that this metadata can be used to reflect types and generate representations for the runnable JavaScript code. This representations may then be used to verify if an object conforms, based on its structure. Below is a type alias declaration of its simplest form in TypeScript:
\begin{JsCode}[numbers=none]
type MyString = string;
let foo: MyString = 10 as any;
\end{JsCode}
This code compiles without any errors, since the number is assigned as \emph{any} type to a variable, which should only accept strings. In the resulting JavaScript program, the type alias, as well as the type assertion, is removed. A concept to keep this declaration also for the compiled code is shown in the following code snippet:
\begin{JsCode}[numbers=none]
const MyString = reflect("string");
let foo = 10;
\end{JsCode}
In this case, the name of the type alias is used as identifier for a variable declaration. Furthermore, the name of the type is used to pass it as a string to a method, which should return a reflection for it. This still results in a number being assigned to a variable, which should only accept strings. To catch this type incompatibility, the final JavaScript code could check the value, that should be assigned to \texttt{foo}:
\begin{JsCode}[numbers=none]
const MyString = reflect("string");
let foo = MyString.accepts(10);
\end{JsCode}
Before the number is assigned, it should be passed to the type representation of \texttt{MyString}, which should check if the received value is compatible. In case of a violation, the program should report an error.

\section{Definition of Cases}
\label{sec:type-check-situations}

The current situation for TypeScript projects, as discussed in Section~\ref{sec:existing-projects}, observed that runtime type checks cannot be generated automatically at this time. Additional effort is required to integrate code safety features for the compiled program. This means that situations may be missed where checks would be of advantage. In order to achieve a development environment where as many undetectable errors (see Sec.~\ref{sec:undetectable-errors}) as possible are reported during execution time, it may be beneficial to automate the inclusion of runtime validations. In order to provide generated runtime type checks for a TypeScript project, cases have to be collected where such checks have to be performed.

\subsection{Interfaces and Type Aliases}

Interface and type alias declarations are removed by the TypeScript compiler and therefore need to be described for the runtime environment. The name of the given type definition should be used to declare a variable, holding all required information to check any value for conformance.

\subsection{Variable Declarations and Assignments}

If a variable was declared it also has a type bound to it during compile time. This type should be used to declare another variable alongside the original declaration, containing the type description or reference. When assigning a value to a variable, type compatibility should be checked by using the type description declaration.

\subsection{Type Assertions}

Type assertions are comparable to type casts in other languages, with the difference that no special checks or conversions are performed (see Sec.~\ref{sec:ts-type-assertions}). To inspect if an assertion is valid, the same checks should be performed as for variable assignments. Values asserted as \emph{any}, as discussed in Section~\ref{sec:undetectable-errors}, can be ignored, as they would always pass.

% As discussed in section~\ref{sec:undetectable-errors}, it is possible to opt out of compile time checks by asserting a value as \emph{any}. To detect situations, 

\subsection{Functions}

There are different types of functions, which need to be distinguished: regular function declarations, function expressions, and arrow functions (see Sec.~\ref{sec:latest-improvements}). For any of these types, the function has to be reflected to enable type comparison. The runtime description has to include its parameters, which can also be optional, and its return type. If the function is called, the parameters passed, as well as the returned value, have to be checked. Additionally, a function can make use of generics to define parameter or return types~\cite{TypeScriptHandbook:Generics}, as shown below:
\begin{JsCode}[numbers=none]
function foo<T>(bar: T): T {
  return bar;
} 
\end{JsCode}
Whatever type the parameter \texttt{bar}---passed to function \texttt{foo}---has, the returned value must be of the same type, as both are annotated with the generic type parameter \texttt{T}.

\subsection{Enums}

Enumerations (see Sec.~\ref{sec:ts-enums}) are compiled to self executing functions, which initialize a corresponding object~\cite{TypeScriptHandbook:Enums} (see Prog.~\ref{prog:enum-compiled}).
\begin{program}
\caption{The enum \texttt{MyEnum \{ A \}} compiled to JavaScript.~\cite{TypeScriptHandbook:Enums}}
\label{prog:enum-compiled}
\begin{JsCode}
var MyEnum;
(function (MyEnum) {
    MyEnum[MyEnum["A"] = 0] = "A";
})(MyEnum || (MyEnum = {}));
\end{JsCode}
\end{program}
To enable type checks for the runtime, the enum has to be described with their members. In the case of Program~\ref{prog:enum-compiled}, it has exactly one member with the value of the number zero.

\subsection{Classes}

For classes a multitude of cases, which require runtime reflection and checks, have to be considered. Most importantly the entire class---including type parameters (i.e., generics), its members, extending classes and implemented interfaces---has to be reflected to use it as type reference at other places of the program. Methods can be checked the same way as functions, with the difference that they may also use class type parameters as type annotations. Furthermore, when instantiating a class, checks should be performed to ensure that it correctly implements its interfaces.

% reflection, methods, instance members, overloading, extending, implementing interfaces, generics

\subsection{Type Queries}

It is possible to use a value's type as type annotation in TypeScript, which looks like the following:
\begin{JsCode}[numbers=none]
let foo: string = "Bar";
type MyType = typeof foo;
\end{JsCode}
In this case \texttt{MyType} is of type string. TypeScript's \emph{typeof} operator is not to be confused with JavaScript's built in operator of the same name. In TypeScript it is possible to query the type of any identifier, if not attempting to reuse it as a value. In JavaScript, on the other hand, a type query result may be used as value, while it can distinguish between six value types at runtime (see Tab.~\ref{tab:typeof}). If a variable is annotated with a type query, the type of this variable should also be obtainable at runtime.

\subsection{Externals}

JavaScript programs usually make use of other libraries, which are imported alongside 
other project code. If this packages are written in TypeScript, or provide type declaration files, the compiler can use the type information to perform compile time checks. However, as types are also removed from external projects their interfaces, type aliases, class reflections, etc., have to be collected and have to be made available to the runtime code.

\subsection{Ambient Declarations}

If globals are not available in the development environment of a project, but it is known that they will be present in the environment of execution, modules, classes, functions and variables can be declared for the compiler without an implementation:
\begin{JsCode}[numbers=none]
declare function foo(bar: number): string;
\end{JsCode}
After the function \texttt{foo} has been declared as shown above, it can be used according to its signature throughout the project, but it will be removed for the compiled code. Such declaration should be collected and should be made available to the runtime the same way as externals.

\section{Required Steps}
\label{sec:required-steps}

After situations of undetectable errors (see Sec.~\ref{sec:undetectable-errors}) have been clarified, the desired result of the project (see Sec.~\ref{sec:desired-result}) has been outlined, and conditions where transformations should take place (see Sec.~\ref{sec:type-check-situations}) have been pointed out, the steps required to accomplish automated runtime type checks generation are specified:
\begin{enumerate}
  \item Set configurations for the transformation process.
  \item Read the source files of a TypeScript project.
  \item Analyze the source code provided.
  \item Represent the input as an abstract data structure.
  \item Scan the abstraction to obtain type information and relationships.
  \item Perform static type analysis and checks.
  \item Insert runtime type reflections and assertions.
  \item Emit target code for the JavaScript runtime engine.
\end{enumerate}
These steps are described in more detail in the following sections. While giving a theoretical understanding of the concept of the project, no technical details are provided at this point.

\subsection{Configuration}
\label{sec:steps-configuration}

As different projects have different requirements regarding the result of the JavaScript target code, configurations for the transformation process have to be set in advance. This includes the settings for the TypeScript compiler itself\footnote{https://www.typescriptlang.org/docs/handbook/compiler-options.html}, as well as adjustments for \emph{ts-runtime}. While the project of this thesis is not intended to be a replacement of the TypeScript compiler, it should still honor the options of the development environment. This settings include, among others, the ECMAScript version of the resulting program, the module system to use, as well as the write location of the output~\cite{TypeScriptHandbook:CompilerOptions}.

\subsection{Read Source Files}

The starting point of the transformation process should be an existing project. As for a usual TypeScript compilation process, the entry files should be passed to \emph{ts-runtime}, alongside a set of configurations. All files that are referenced or imported throughout the project should be loaded, resulting in a reflection of the project's file system structure. This should enable further steps to interact with the input in memory, leaving the original files untouched.

\subsection{Syntax Analyzation}

After all contents of the project are available it should be determined, if the provided code is syntactically correct. This should prevent \emph{ts-runtime} to fail, due to syntax errors in the source. If syntactic errors are detected the process should be stopped immediately to prevent the occurrence of unexpected behavior or results.

\subsection{Abstraction}

In order to perform special checks and transformations to the original code, abstracting the source will be beneficial. A suitable data structure may be an abstract syntax tree (i.e., AST), as described in Section~\ref{sec:ast}. Performing modifications on the input directly via string modifications is much more error prone, and semantic connections between parts of the code cannot be extracted easily.

\subsection{Scan Abstraction}

Once the provided source files can be considered syntactically correct and are represented in an abstract data structure, the type information has to be extracted for future modifications to the code. It should not only be possible to obtain the explicitly set type of an AST node, but to also receive the implicitly inferred type. In addition it should be practicable to follow a type reference's type, for further processing. To ensure that important data---e.g., type information, AST node relations, and declared identifiers---is not becoming inaccessible during the transformations, the abstract syntax tree may be scanned ahead of changing its nodes.

\subsection{Static Type Checks}

Another important aspect is to already perform static type checks, and to reject the input from further processing, if type incompatibilities can be detected. This has the advantage of flagging issues to developers early. If static type analysis can already find possible violations, the target code may not behave correctly. Anyway, as it is possible to provide incorrect type declarations for accurate implementations, there should be the possibility to force the process to proceed and to solely rely on type compatibility checks at runtime. Warnings should be generated at compile time in this case, to clearly indicate that unexpected results may be a consequence.

\subsection{Transformations}

Situations where modification have to take place to reflect all required type information, as well as to introduce runtime type checks based on these reflections, have already been identified in Section~\ref{sec:type-check-situations}. All required data to perform extensive transformations on the AST should have already been prepared by the previous steps of the process. This should enable \emph{ts-runtime} to proceed with substituting and altering abstract syntax tree nodes. The tree should be scanned from the bottom to the top, which should guarantee that transformations of high level tree nodes already include low level mutations. Furthermore, a node that is replaced or changed should be mapped to the original syntax tree node to assure that its initial state can be retrieved at any time.

\subsection{Target Code Emit}

After the transformations have been applied to the syntax tree, it should be converted to TypeScript compatible syntax code. In a next step this code can then be used to emit runnable JavaScript, according to the options that were passed when initially triggering the generation of runtime type checks (see Sec.~\ref{sec:steps-configuration}). In case of inconsistencies or faults, appropriate warnings and errors should be triggered.

\section{Summary}
\label{sec:theoretical-approach-summary}

To achieve automatically generated runtime type reflections and checks, a series of steps have to be carried out. To give a better understanding of the conceptual procedure, Figure~\ref{fig:theoretical-approach} illustrates the idea of the transformation process. In the following chapter the theoretical approach is evaluated, and the program technology and architecture is defined. Also technical peculiarities and limitations are identified to provide a solid base for the implementation.
\begin{figure}
\centering
\includestandalone{assets/diagrams/theoretical-approach}
\caption{Conceptual procedure of applying transformations to a TypeScript project.}
\label{fig:theoretical-approach}
\end{figure}

% honor TS compiler options
% drop in replacement
% read
% analyze
% get type info
% modify/transform
% write
% verify

%\section{Concept}
%\label{sec:concept}
%
%After having clarified the current situation and pointed out conditions where type reflections and assertions should take place, a concept for the actual project---further referred to as \emph{ts-runtime} (i.e., \emph{typescript-runtime}) or \emph{tsr}---can be elaborated. This includes an identification of the steps required to achieve the desired result of automated runtime type check generation. Also various parts of the program will already be suggested.

%\subsection{Obtaining Type Information}
%\label{sec:obtaining-type-information}

%\subsection{Components}

% Itemize
% Components figure

% honor TS compiler options
% drop in replacement

% read
% analyze
% get type info
% modify/transform
% write
% verify

%\subsubsection{Transformer}
%
%\subsubsection{Event Bus}
%
%\subsubsection{Scanner}
%
%\subsubsection{Factory}
%
%\subsubsection{Mutators}
%
%\subsection{Runtime Library}
%
%\subsection{Executable Binary}

%\subsection{Procedure}

%\subsection{Operation}
%
%\subsubsection{Configuration}
%
%\subsubsection{Initialization}
%
%\subsubsection{Procedure}

%\subsection{Components}
%
%\subsection{Procedure}
%
%\subsection{Operation}



% how to acheive the desired result

%\subsection{Type Check Situations}
%\label{sec:type-check-situations}

%\section{Type Check Situations}
%\label{sec:type-check-situations}
%
%\subsection{Main Cases}
%\label{sec:main-cases}
%
%\subsection{Edge Cases}
%\label{sec:edge-cases}
%
%\subsection{Exceptions}
%\label{sec:exceptions}
%
%\subsection{Procedure}
%\label{sec:procedure}