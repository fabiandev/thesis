\chapter{Summary and Outlook}
\label{cha:summary_outlook}

This thesis explored the field of runtime type checks for JavaScript, with a detailed overview of its type system, which was also compared to those of other programming languages. Following, the JavaScript superset TypeScript was examined in detail to provide a sophisticated overview of its characteristics and features, while also pointing out differences and similarities with the superset Flow. It could be determined that the static compile time type analysis of TypeScript can detect a multitude of potential errors, while there are also situations where the TypeScript compiler cannot detect possible issues for the target code. As no additional type checking techniques are included and type informations are not available in the compiled JavaScript code, other techniques have to be employed to ensure that unexpected conditions can be observed and reacted to during program execution, resulting in increased development and maintenance effort. Therefore a method was developed to automatically generate runtime type checks based on the type annotations of a TypeScript project. A theoretical concept was constructed, before a project was implemented that can extract the required information and to emit a JavaScript program with integrated runtime validations. For the runtime type system itself a third party library was utilized, which was developed with the JavaScript superset Flow in mind, but provides a multitude of features which are applicable for runtime type checks that align with the behavior of TypeScript's static type system.

Subsequently the resulting project was evaluated---including its API, CLI and the runnable JavaScript code---to prove its quality and functionality and to also provide insights into performance analyzations and benchmarks. The findings verified the operability of the project of this thesis, and the functioning of the target code. Build times are in an acceptable range, compared to the native TypeScript compiler, and type incompatibilities are reported during runtime. While the the generation of type checks is efficient, the performance of the produced code at runtime cannot compete with the unmodified version. This is attributable to the comprehensive type system that is included, as well as its internal verification processes. While a decrease in execution time was to be expected, the results were not satisfactory to be used in a production system. However, making use of the thesis project in the phase of development may be beneficial to observe unexpected values at execution time. This suggests that to further improve the performance at runtime, a different library should be utilized, which carries out type checks more efficiently. An interface could be provided which supports exchanging the runtime type system, without the requirement to make changes to the underlying program. This would also allow other developers to make use of a custom implementation or a preferred library, with as little effort as possible.

To ensure that future development does not break operability, an extensive collection of automated unit tests and code coverage statistics are part of the project to also enable continuous integration, which should preserve code quality over time and report unexpected behavior that may be introduced with changes to the code base.

% evaluation (good parts, bad parts: flow-runtime)
% next steps (transition to io-ts)

