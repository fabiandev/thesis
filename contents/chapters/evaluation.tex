
\chapter{Evaluation}
\label{cha:evaluation}

In this chapter, the implemented project of this thesis will be tested and evaluated. Different methods will be used to measure and verify the quality and functionality of \emph{ts-runtime}.

\section{Automated Unit Tests}
\label{sec:automated-unit-tests}

To ensure the correctness of the source code transformations applied to a TypeScript project, a series of unit tests\footnote{``A unit test generally exercises the functionality of the smallest possible unit of code (which could be a method, class, or component) in a repeatable way~\cite{UnitTests:Android}'' to ''[...] verify that the logic of individual units is correct.~\cite{UnitTests:Android}''} are provided, which should be run after modifications were made to the project's source code. These tests should raise errors, if a mutation changes unexpectedly, possibly resulting in wrong behavior when utilized at runtime. If such a change is intended, the corresponding tests have to be updated as well.

\section{Performance Analyzation}
\label{sec:performance-analyzation}



\section{Comparison}
\label{sec:evalutaion-comparison}

%\subsection{Javascript without Type Checks}
%\label{sec:javascript-without-type-checks}
%
%\subsection{Javascript with Manual Type Checks}
%\label{sec:javascript-with-manual-type-checks}
%
%\subsection{Flow with Generated Type Checks}
%\label{sec:flow-with-generated-type-checks}

\section{Interpretation and Conclusion}
\label{sec:interpretation-conclusion}
